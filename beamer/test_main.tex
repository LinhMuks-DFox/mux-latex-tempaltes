\documentclass[aspectratio=169, 12pt]{beamer}
\usetheme{mux_default_presentation}

% ==========================================
%  语言与字体模式选择 (Language Modes)
%  请只保留一种模式,注释掉其他模式
% ==========================================

% ---【模式 A: 中文 (Chinese)】---
% 自动调用系统默认无衬线中文字体 (Mac: 苹方/STHeiti, Win: 微软雅黑)
\usepackage[UTF8, heading=false]{ctex} 

% ---【模式 B: 日文 (Japanese)】---
% 需注释掉上面的 ctex,启用下面的 xeCJK
% \usepackage{xeCJK}
% \setCJKmainfont{Hiragino Kaku Gothic ProN} % Mac 推荐: 冬青黑体
% % \setCJKmainfont{Yu Gothic}               % Win 推荐: 游黑体
% \setCJKsansfont{Hiragino Kaku Gothic ProN} % 强制无衬线

% ---【模式 C: 英文 (English)】---
% 不需要 ctex 或 xeCJK,建议手动指定 Helvetica 以确保 Keynote 风格
% \usepackage{fontspec}
% \setmainfont{Helvetica}
% \setsansfont{Helvetica}

% ==========================================

% 强制英文字体使用无衬线 (配合上述设置)
\renewcommand{\familydefault}{\sfdefault} 

% --- 元数据 ---
\title{Mux Presentation}
\subtitle{Multilingual Template}
\author{Mux}
\date{\today}

\begin{document}
% ... (后续内容保持不变)
\begin{document}

% --- 1. 封面 (自动右对齐) ---
\begin{frame}
    \titlepage
\end{frame}

% --- 2. 列表层级展示 ---
\begin{frame}{逻辑层级 (Hierarchy)}
    \framesubtitle{测试不同层级的符号辨识度}

    Beamer 列表系统已被修改,以确保最大可读性:

    \vspace{1em}

    \begin{itemize}
        \item \textbf{一级标题 (实心圆点)}:这是最主要的观点。
        \item 依然是一级标题,保持整齐。
              \begin{itemize}
                  \item \textbf{二级标题 (对勾)}:表示确认、完成或子集。
                  \item 符号 `checkmark` 具有强烈的视觉引导性。
                        \begin{itemize}
                            \item 三级标题 (方块):用于补充细节参数。
                            \item 细节不应喧宾夺主。
                        \end{itemize}
              \end{itemize}
        \item 回到一级标题。
    \end{itemize}
\end{frame}

% --- 3. 盒子设计 (Box Design) ---
\begin{frame}{强调区域 (Blocks)}
    \framesubtitle{用于展示定义、定理或重要提示}

    \begin{columns}[t] % 顶部对齐
        \column{0.48\textwidth}
        % 普通盒子
        \begin{block}{常规设计 (Standard Block)}
            这是最基础的盒子。\\
            黑底白字标题 + 淡灰背景。\\
            适合放置:定义、核心概念。
        \end{block}

        \vspace{1em}

        % 无标题盒子(默认只有灰色背景)
        \begin{block}{}
            这是一个没有标题的 Block,仅作为灰色背景板使用,适合引用一句名言。
        \end{block}

        \column{0.48\textwidth}
        % 警告/强调盒子
        \begin{alertblock}{注意 (Alert Block)}
            这是用于强调风险或错误的盒子。\\
            颜色自动变为红色系,引起观众警觉。
        \end{alertblock}

        \vspace{1em}

        % 示例盒子
        \begin{exampleblock}{示例 (Example Block)}
            Beamer 默认的绿色系,适合放代码片段或成功案例。
        \end{exampleblock}
    \end{columns}
\end{frame}

% --- 4. 图文混排 (Layout) - 修复版 ---
\begin{frame}{图文排版 (Layout)}
    \framesubtitle{左右分栏是 Keynote 最经典的用法}

    \begin{columns}
        % 左侧栏
        \begin{column}{0.6\textwidth}
            \textbf{排版建议:}
            \begin{itemize}
                \item 左侧放要点,右侧放图(或反之)。
                \item 图片宽度建议设为 \texttt{\textbackslash linewidth}。
                \item 尽量保持留白,不要填满整个页面。
            \end{itemize}

            \vspace{1em}
            \begin{block}{图片说明}
                使用 columns 环境可以轻松分割页面。
            \end{block}
        \end{column}

        % 右侧栏
        \begin{column}{0.35\textwidth}
            \centering
            % 这里的 rule 是个占位符,实际使用时换成 \includegraphics[width=\linewidth]{图片.png}
            \rule{\linewidth}{5cm}
            \vspace{0.5em}
            \par
            {\tiny 图 1: 占位符示意图}
        \end{column}
    \end{columns}
\end{frame}

% --- 5. 结束页 ---
\begin{frame}
    \centering
    \vspace{0.2\paperheight}
    {\Huge \textbf{Thank You}}\par
    \vspace{1em}
    Q \& A
\end{frame}

\end{document}